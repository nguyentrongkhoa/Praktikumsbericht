\chapter{Reflexion}

Insgesamt lässt sich sagen, dass die Praktikumszeit bei DesCap für mich sehr gelungen war. Dadurch, dass DesCap ein Startup und kein 
Großunternehmen ist, konnte ich von einer recht planaren Unternehmensstruktur zeugen. Das hatte für mich den Vorteil,
dass alle mit allen was zu tun haben und man nicht auf einen repetitiven Aufgabenbereich eingeschränkt ist, sondern dynamisch
bei vielen interessanten Tätigkeiten eingebunden wird. 

Was den Inhalt und Umfang der Aufgabenstellungen anbelangt, bin ich auch sehr zufrieden mit dem Gelerten bzw. Geleisteten. 
Viel von dem erworbenen Wissen, vor allem wenn es um das PCB-Design, den Umgang mit Mikrokontrollern, 3D-Druck, oder die Datenerfassung 
und -auswertung mit Sensoren sind definitiv höchstrelevant für die Robotik. 

Eine weitere Sache, die ich gelernt habe und die ich auch für erheblich halte, ist das Kollaborieren im Team. Anfangs musste ich mir
viele "Hobbyisten-Gewohnheiten" abgewöhnen und mich daran erinnern, dass fehlende Dokumentation nichts Gutes bedeutet, denn dadurch 
werden vermeidbare Fallen und Fehler von anderen Teammitgliedern, die sich anderen Themen widmen. Diese Mentatilät habe ich seither
also für persönliche Projekte adoptioniert, nämlich dass bei jedem Projekt sofort eine Github-Repo einzurichten ist mit einer
ausführlichen README-Datei und Literatur über das Thema. 

Apropos Hobbyisten vs professionell wurde mir klar, wie wichtig eine Strukturierung im Voraus ist. Hobbyisten-Projekte sind ja
meistens kurzfristig, in sich selbst geschlossen und haben ein klar definiertes Endziel, weswegen ein direkt-loslegen-Ansatz 
selten Konsequenzen hat. Unternehmensprojekte sind hingegen offen, denn ein Produkt muss sich immer weiterentwickeln, um sich 
dem Markt und der Konkurrenz anzupassen. Außerdem gibt es Fristen einzuhalten, und ohne klare Struktur am Anfang würde man 
wertvolle Zeit verlieren. 

Abschließend würden sich meiner Meinung nach ein paar Gedanken zur zwischenmenschlichen Ebene bei DesCap lohnen. Ebenfalls in dieser Hinsicht 
hat mir das Praktikum bei DesCap sehr gefallen. Meine Kollegen waren besonders hilfsbereit und haben mich bei jeglicher 
Schwierigkeit gern unterstützt. Um mit einem Sprichwort alles zusammenzufassen: Ende gut, Anfang gut, alles gut. 