\chapter{Praktikumstätigkeiten}

\section{Hardware-Design}

Meine Hauptaufgabe bei DesCap besteht in dem Designen und Validieren von PCBs für das Produkt "Work",
welches zum Beispiel vor stäubigen Umgebungen bei handwerklichen Arbeiten schützt. Auf der Leiterplatte
befinden sich unter anderem ein DC-Spannungswandler (zum gleichzeitgen Betreiben von zwei Lüftern) und
ein MSP430 Mikrokontroller, mit dem sich alle Nutzerfunktionen realisieren lassen. 

Schon bei der Einarbeitungsphase wurde darauf hingewiesen, dass ein komplett neues PCB-Design gemacht werden musste aufgrund des
Mikrokontrollerwechsels (ATTiny zu MSP430). In den ersten zwei Wochen bekam ich die Aufgabe, mich mit der Schaltung des Vordesigns 
vertraut zu machen, wofür es enge Betreuung und Hilfestellung gab in Form von täglichen Treffen mit dem Engineering-Team. Dazu kamen
auch kleinere Hardware-Recherchen, um etwa den richtigen Knopf oder die kostengünstigsten LEDs für das PCB zu finden. 

Da ich im Vorfeld leider kein KiCad-Wissen verfügt habe, habe ich mich für 2 bis 3 Wochen im Anschluss mit den Grundlagen des PCB-Designs
sowie der Bedienung von CAD-Softwares wie KiCad auseinandergesetzt. Das eigentliche Design hat 2 Wochen beansprucht, und während dieser Zeit
gab es ständig Austausche, denn Christopher konnte sich über Github meinen Fortschritt ansehen und mir ggfs Rückmeldung geben.
Da das neue Board über eine USB-Schnittstelle flashbar sein sollte und diese strenge Anforderungen bezüglich Routens und Impedanzmatchings
stellte, haben wir uns zusätzlich noch externe Expertise für ein ausführliches Design-Review geholt. 

Dabei habe ich gelernt, wie man mit welchen Tools bspw. die Impedanz eines USB-Diff-Pairs berechnen kann, da es der kritischste Aspekt war.
Für dieses alleerste Design kamen wir auf 86$\Omega$, also eine akteptable Abweichung vom Idealwert 90$\Omega$. Bei dem Review wurde außerdem
der für uns sehr wichtige Gesichtspunkt EMV (elektromagnetische Verträglichkeit) besprochen zwecks Zertifizierung. Es wurden die "best practices"
erklärt, wie man unter anderem DC-DC-Wandler routen oder welche Glättungskondensatoren man dafür nehmen soll.

Der nächste Schritt war der Bestellungsablauf. Hier habe ich mich an erster Stelle um die Auswahl von Komponenten gekümmert:
geläufige Bauteile wie Widerstände oder Kondensatoren wurden nach ihren Werten optimiert (BOM-Konsolidierung); bei anderen Bauteilen
wie Knöpfen, Schaltern oder LEDs wurde auf Kosten und langfristige Verfügbarkeit geachtet, damit zukünftig zeitkritischen großen Bestellungen
nicht am Supply-chain scheitern.

Eine Woche etwa nach Bestellungsabschluss kamen die Leiterplatten von China bei uns in Berlin an. Das Flashen vom MSP430, welches
zur Erinnerung am meisten Kopfschmerzen bereitet hat, hat entgegen Erwartungen auf Anhieb funktioniert. Wir hätten gedacht, dass wir 
ein paar Design-Runden bräuchten, um das halbwegs auf die Reihe zu bekommen. Christopher war demgemäß sehr glücklich  und hat infolgedessen
alle zum Mittagsessen in der Schindler-Mensa eingeladen. 

Nach diesem enormen Meilenstein bestanden weitere Entwicklungsarbeiten, zumindest was das PCB-Design anging, nur noch in kleinen 
Modifikationen von einer Version zur anderen: manche Komponenten wurden gegen bessere und günstigere ausgetauscht, falsche Slkscreen-Texte wurden
berichtigt usw. Lediglich bei der fünften Iteration ist eine Herausforderung aufgetreten, nämlich dass der Bestand für unseren USB-Stecker
bei JLCPCB zu gering war und die Nachlieferung unsere Zeiteinschränkungen überschritten hat. Nach tagelanger Recherche wurde für eine
nachhaltigere Variante entschieden. Allerdings musste ich selber den Footprint davon erstellen, da er nicht in der Standard-KiCad-Library
vorhanden war. Aufgrund dessen, dass man im Kicads Footprint-Editor bei komplexen Geometrien im Gegensatz zu Solidworks z.B die Maßen und Abstände der Pads nicht
genau in Verhältnis zueinander setzen kann, muss über Umwege gegangen werden. Letztendlich musste ich die mechanische Zeichnung vom 
Footprint in Solidworks umsetzen, diese im Anschluss mit Inkscape bearbeiten, um sie endlich als Vorlage in Kicad zu importieren.

% TODO: add foto of D+ and D-
Die Schwierigkeiten hören noch nicht auf. Die Pads des neuen USB-Steckers sind viel dichter bevölkert als beim alten, und das hat für
einen veritablen Kampf gesorgt. Weil unser Stecker ein vertikaler und kein horizontaler ist wie die meisten USB-Stecker, müssen 
die D+ und D- Differentialleitungen in dem ganz schmalen Zwischenraum des Footprints überkreuzt verbunden werden. Dies erfordert,
dass ein Via dazwischengequetscht werden muss, wodurch jedoch die DFM-Kapabilitäten (design for manufacturing) von unserem 
PCB-Produzenten JLCPCB regelrecht auf die Probe gesetzt werden. 

Für das Produkt fungiert der oben erwähnte MSP430 als Funktionszentrale, und dieser muss natürlich auch programmiert werden.
An dem Programmierprozess habe ich mich nicht direkt beteiligt, denn sonst wären mir die Aufgaben über den Kopf gewachsen.
Nichtsdestoweniger musste ich mir die Datenblätter davon durchlesen, um am Ende den Code nachvollziehen zu können. Sowohl beim 
PCB-Design als auch beim Programmieren werfen mindestens 2 Leute einen Blick rein, um die Fehlerwahrscheinlichkeit zu minimieren.
Meinen Beitrag zur MSP-Thematik habe ich nicht unbedingt auf C-Code-Ebene geleistet, sondern vielmehr auf Assembler-Ebene, denn ich 
musste etliche Schutzfunktionen implementieren und es ging nur, wenn man direkt den Linker-Skript modifiziert. Obwohl nicht notwendig,
habe ich mich deswegen auch darin vertieft, wie Linker und Loader hinter der Kulisse funktionieren.

Abgesehen vom Hauptprodukt habe ich auch andere Designprojekte aufgenommen. Eines davon war ein Breakout-Board, das Partikelsensoren
mit einem ESP32 verbindet. Dieses sollte Bestandteil von unserem Teststand zum Messen der Filtereffizienz, allerdings wurde es nicht
in Betrieb genommen, weil wir uns anders überlegt haben.

Ein weiteres Projekt, an dem ich mich seit Kurzem beteilige, ist die Integration von Sensoren in das "Work"-Produkt, die in der Lage
sein sollen, sich untereinander mithilfe einer Basisstation über das LoRaWan-Protokoll Daten bezüglich der Innenraumluftqualität
auszutauschen. Dies soll dem Zweck dienen, Simulationsergebnisse mit der wirklichen Betriebsfiltereffizienz zu vergleichen sowie 
Nutzer rechtzeitig warnen, sobald Grenzwerte überschritten werden. Dieses Projekt hat mir DesCap als Thema für die Bachelorarbeit vorgeschlagen
und sieht vorerst folgende Unteraufgaben vor: Definieren von Anforderungen DesCaps (bzgl. Datenrate, Reichweite usw), Recherche und
Auswahl eines geeigneten Mikrokontrollers bzw Entwicklungsboards, PCB-Prototypen entwerfen \dots

\section{Hardware-Testen}

Da es sich um Verbraucherprodukte handelt, ist das Testen genauso wichtig, wenn nicht sogar wichtiger, als das Designen von Leiterplatten.
Hiermit verbringe ich also viel Zeit bei DesCap, um sicherzustellen, dass alles zuverlässig funktioniert und alle Missbrauchsszenarien
seitens des Nutzers abgedeckt werden. 
% TODO: add Testanforderungen on github as picture
Dafür haben wir uns schon nach Bestellungsabgabe der PCBs eine umfangreiche Teststrategie überlegt. Dazu gehören unter anderem folgende Punkte:
\begin{enumerate}
    \item Welche Größen von welchen Komponenten müssen erfasst werden? Z.B. Temperatur und Entladespannung der Batterie, PWM-Duty-Cycle der Lüfter, Ladestrom \dots
    \item Zulässige Wertebereiche für gemessene Signale
    \item Durchscnittliche Standzeit in verschiedenen Betriebsmodi
\end{enumerate}

Bei der ersten Batch wurde alles noch manuell erfasst, das heißt die PCBs enthielten zahlreiche Testpads, die beim Bedarf gelötet
und an das Messgerät herangeführt werden mussten. Schnell haben wir festgestellt, dass es zwar funktioniert hat, aber recht mühsam war,
wenn man mehrere Boards hintereinander testen wollte. Mit dieser Erkenntnis habe ich mir zur Aufgabe gemacht, einen kostengünstigen
\href{https://github.com/nguyentrongkhoa/PCB_test_jig?tab=readme-ov-file#tests}{PCB-Test-Jig} zu bauen. 

Der Aufbau entspricht einem typischen kommerziellen Testjig, mit kleinen Anpassungen: die Pogopins, die die SMT-Testpads auf dem PCB
berühren sollen, werden auf einem Perfboard gelötet und auf der Unterseite des 3D-gedruckten PCB-Trägers angebracht. Die Oberseite
hat einen Ausschnitt mit der gleichen Geometrie wie das PCB selbst. Die Pogopins werden auf der anderen Seite über Drähte mit dem Messgerät verbunden
(in diesem Fall der Analog Discovery 3). 

\begin{figure}[H]
     \centering
     \begin{subfigure}[b]{0.48\textwidth}
         \centering
         \includegraphics[width=\textwidth]{kapitel2/pictures/pogo_1.jpg} % Name deines ersten Bildes
         \caption{Vorderansicht}
         \label{fig:bild1}
     \end{subfigure}
     \hfill % Erzeugt den horizontalen Abstand zwischen den Bildern
     \begin{subfigure}[b]{0.48\textwidth}
         \centering
         \includegraphics[width=\textwidth]{pogo_2} % Name deines zweiten Bildes
         \caption{Seitenansicht}
         \label{fig:bild2}
     \end{subfigure}
     \caption{Pogo-Pins im PCB-Test-Jig}
     \label{fig:gesamtbild}
\end{figure}

\begin{figure}[H]
     \centering
     \begin{subfigure}[b]{0.48\textwidth}
         \centering
         \includegraphics[width=\textwidth]{pogo_pins} % Name deines ersten Bildes
         \caption{Oberteil}
         \label{fig:bild3}
     \end{subfigure}
     \hfill % Erzeugt den horizontalen Abstand zwischen den Bildern
     \begin{subfigure}[b]{0.48\textwidth}
         \centering
         \includegraphics[width=\textwidth]{print_2_parts} % Name deines zweiten Bildes
         \caption{Unterteil}
         \label{fig:bild4}
     \end{subfigure}
     \caption{3D-gedrucktes Jig in Solidworks}
     \label{fig:gesamtbild1}
\end{figure}

\begin{figure}[H]
     \centering
     \begin{subfigure}[b]{0.48\textwidth}
         \centering
         \includegraphics[width=\textwidth]{test_jig_top} % Name deines ersten Bildes
         \caption{Vorderansicht}
         \label{fig:bild5}
     \end{subfigure}
     \hfill % Erzeugt den horizontalen Abstand zwischen den Bildern
     \begin{subfigure}[b]{0.48\textwidth}
         \centering
         \includegraphics[width=\textwidth]{test_jig_overview} % Name deines zweiten Bildes
         \caption{Seitenansicht}
         \label{fig:bild6}
     \end{subfigure}
     \caption{Gesamter Aufbau}
     \label{fig:gesamtbild2}
\end{figure}

Auf diesem Wege können sehr aussagekräftige Kennlinien aufgezeichnet werden: 

\begin{figure}[H]
     \centering
     \begin{subfigure}[b]{0.48\textwidth}
         \centering
         \includegraphics[width=\textwidth]{test_5} % Name deines ersten Bildes
         \caption{Bei hoher Lüfter-PWM-Zahl}
         \label{fig:bild7}
     \end{subfigure}
     \hfill % Erzeugt den horizontalen Abstand zwischen den Bildern
     \begin{subfigure}[b]{0.48\textwidth}
         \centering
         \includegraphics[width=\textwidth]{test_7} % Name deines zweiten Bildes
         \caption{Bei hoher Lüfter-PWM-Zahl}
         \label{fig:bild8}
     \end{subfigure}
     \caption{Batterieentladung bei verschiedenen Funktionsmodi}
     \label{fig:gesamtbild3}
\end{figure}

\begin{figure}[H]
     \centering
     \begin{subfigure}[b]{0.48\textwidth}
         \centering
         \includegraphics[width=\textwidth]{test_6} % Name deines ersten Bildes
         %\caption{Bei hoher Lüfter-PWM-Zahl}
         \label{fig:bild9}
     \end{subfigure}
     \hfill % Erzeugt den horizontalen Abstand zwischen den Bildern
     \begin{subfigure}[b]{0.48\textwidth}
         \centering
         \includegraphics[width=\textwidth]{test_8} % Name deines zweiten Bildes
         %\caption{Bei hoher Lüfter-PWM-Zahl}
         \label{fig:bild10}
     \end{subfigure}
     \caption{Batterieaufladung}
     \label{fig:gesamtbild4}
\end{figure}

Damit jegliche Entwicklungsschritte auch im Nachhinein nachvollziehbar sind, wird alles möglichst
detailliert zu dokumentieren. Ich lege großen Wert darauf, nach Beenden jeder kleinen Aufgabe dies zu erledigen, bevor ich zur 
nächsten übergehe. 

\section{Filter}

Da DesCap wie erwähnt sich mit tragbaren Atemschutzgeräten befasst, kommt man nicht um das Thema Filter herum, und genau das stellt einen weiteren wichtigen Tätigkeitsbereich für mich dar. 
Dabei setze ich mich hauptsächlich mit drei Richtungen der DesCap-Filterstrategie auseinander.

\subsection{Mechanische Partikelfilter}

An erster Stelle stehen rein mechanische Filter für die Partikelgrößen PM$_{10}$ und PM$_{2.5}$ (alle Angaben in $\mu$m). 
In diesem Bereich beschäftigte ich mich zunächst mit den physikalischen Grundlagen der Partikelabscheidung. Mechanische Filter 
arbeiten im Wesentlichen nach dem Prinzip der Siebwirkung, der Trägheitsabscheidung, der Diffusion sowie der Interzeption. 
Abhängig von der Partikelgröße und der Strömungsgeschwindigkeit dominieren unterschiedliche Abscheidemechanismen. 
Während größere Partikel vor allem durch Trägheit und direkte Anlagerung an die Fasern abgeschieden werden, 
spielen bei kleineren Partikeln zusätzlich Diffusionseffekte eine wichtige Rolle.

Im Rahmen meines Praktikums analysierte ich verschiedene Filtermedien hinsichtlich ihrer Porenstruktur, 
Faserdurchmesser und Schichtdicke. Ein zentraler Aspekt war dabei stets das Verhältnis zwischen Filtrationseffizienz 
und Atemwiderstand. Besonders bei tragbaren Atemschutzgeräten ist es entscheidend, einen möglichst hohen Abscheidegrad 
zu erreichen, ohne den Luftdurchfluss zu stark zu behindern. Ich unterstützte bei der Auswertung von Messdaten, 
in denen Druckverlust und Abscheideleistung experimentell bestimmt wurden. Zudem verglich ich unterschiedliche 
Materialkombinationen und deren Einfluss auf Gewicht, Baugröße und Austauschintervalle der Filter.

\subsection{Geruchsfilter}

Der zweite Schwerpunkt lag bei Geruchsfiltern. Im Gegensatz zu mechanischen Filtern geht es hier nicht 
primär um feste Partikel, sondern um gasförmige Stoffe und flüchtige organische Verbindungen. Die Filterwirkung basiert meist 
auf Adsorption, insbesondere durch Aktivkohle oder andere hochporöse Materialien mit großer innerer Oberfläche.

Ich beschäftigte mich mit den Grundlagen der Adsorptionsprozesse und deren Abhängigkeit von Faktoren wie Temperatur, 
Luftfeuchtigkeit und Konzentration der Schadstoffe. Dabei lernte ich, dass die Leistungsfähigkeit eines Geruchsfilters 
stark von der Sättigung des Adsorptionsmaterials abhängt. Ein wichtiger Aspekt war daher die Abschätzung der Standzeit 
solcher Filter unter realistischen Einsatzbedingungen.

Im Rahmen meiner Aufgaben unterstützte ich bei der Bewertung verschiedener Aktivkohlematerialien hinsichtlich ihrer 
spezifischen Oberfläche und Adsorptionskapazität. Außerdem analysierte ich, wie sich die Integration eines 
Geruchsfilters in ein bestehendes Atemschutzsystem konstruktiv auswirkt, beispielsweise in Bezug auf Bauraum, 
Gewicht und Austauschbarkeit. Ziel ist es, eine möglichst kompakte Lösung zu entwickeln, die dennoch eine effektive 
Reduktion unangenehmer oder gesundheitsschädlicher Gase gewährleistet.

\subsection{Elektrostatische Filter}

Der dritte Themenbereich umfasste elektrostatische Filterkonzepte. Diese nutzen elektrische Ladungen, um Partikel aus der 
Luft zu entfernen. Dabei werden entweder die Partikel selbst ionisiert oder das Filtermedium ist elektrostatisch geladen, 
sodass die Partikel durch Coulomb-Kräfte angezogen und gebunden werden.

Ich habe Fachliteratur zu dem Thema Elektretmaterialien gelesen und theoretische Einblicke in deren Auf- sowie Entladungsmechanismen erworben.
Mit dem Wissen wurde anschließend untersucht, inwiefern diese Technologie zur Effizienzsteigerung bei gleichbleibendem oder reduziertem Atemwiderstand beitragen kann. 
Kernpunkt war auch die Gegenüberstellung von rein mechanischen Filtern und Kombinationsfiltern. Hier haben wir uns Fragen gewidmet wie 
"welche Kombination sowie Anordnung der 3 Filtersorten wäre optimal?" bzw. "würde die gesteigerte Filterleistung und -standzeit den 
höheren Produktpreis rechtfertigen?".  
