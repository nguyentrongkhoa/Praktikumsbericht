\chapter{Praktikumstätigkeiten}

\section{Hardware-Design}

Meine Hauptaufgabe bei DesCap besteht in dem Designen und Validieren von PCBs für das Produkt "Work",
welches zum Beispiel vor stäubigen Umgebungen bei handwerklichen Arbeiten schützt. Auf der Leiterplatte
befinden sich unter anderem ein DC-Spannungswandler (zum gleichzeitgen Betreiben von zwei Lüftern) und
ein MSP430 Mikrokontroller, mit dem sich alle Nutzerfunktionen realisieren lassen. 

Schon bei der Einarbeitungsphase wurde darauf hingewiesen, dass ein komplett neues PCB-Design gemacht werden musste aufgrund des
Mikrokontrollerwechsels (ATTiny zu MSP430). In den ersten zwei Wochen bekam ich die Aufgabe, mich mit der Schaltung des Vordesigns 
vertraut zu machen, wofür es enge Betreuung und Hilfestellung gab in Form von täglichen Treffen mit dem Engineering-Team. Dazu kamen
auch kleinere Hardware-Recherchen, um etwa den richtigen Knopf oder die kostengünstigsten LEDs für das PCB zu finden. 

Da ich im Vorfeld leider kein KiCad-Wissen verfügt habe, habe ich mich für 2 bis 3 Wochen im Anschluss mit den Grundlagen des PCB-Designs
sowie der Bedienung von CAD-Softwares wie KiCad auseinandergesetzt. Das eigentliche Design hat 2 Wochen beansprucht, und während dieser Zeit
gab es ständig Austausche, denn Christopher konnte sich über Github meinen Fortschritt ansehen und mir ggfs Rückmeldung geben.
Da das neue Board über eine USB-Schnittstelle flashbar sein sollte und diese strenge Anforderungen bezüglich Routens und Impedanzmatchings
stellte, haben wir uns zusätzlich noch externe Expertise für ein ausführliches Design-Review geholt. 

Dabei habe ich gelernt, wie man mit welchen Tools bspw. die Impedanz eines USB-Diff-Pairs berechnen kann, da es der kritischste Aspekt war.
Für dieses alleerste Design kamen wir auf 86$\Omega$, also eine akteptable Abweichung vom Idealwert 90$\Omega$. Bei dem Review wurde außerdem
der für uns sehr wichtige Gesichtspunkt EMV (elektromagnetische Verträglichkeit) besprochen zwecks Zertifizierung. Es wurden die "best practices"
erklärt, wie man unter anderem DC-DC-Wandler routen oder welche Glättungskondensatoren man dafür nehmen soll.

Der nächste Schritt war der Bestellungsablauf. Hier habe ich mich an erster Stelle um die Auswahl von Komponenten gekümmert:
geläufige Bauteile wie Widerstände oder Kondensatoren wurden nach ihren Werten optimiert (BOM-Konsolidierung); bei anderen Bauteilen
wie Knöpfen, Schaltern oder LEDs wurde auf Kosten und langfristige Verfügbarkeit geachtet, damit zukünftig zeitkritischen großen Bestellungen
nicht am Supply-chain scheitern.

Eine Woche etwa nach Bestellungsabschluss kamen die Leiterplatten von China bei uns in Berlin an. Das Flashen vom MSP430, welches
zur Erinnerung am meisten Kopfschmerzen bereitet hat, hat entgegen Erwartungen auf Anhieb funktioniert. Wir hätten gedacht, dass wir 
ein paar Design-Runden bräuchten, um das halbwegs auf die Reihe zu bekommen. Christopher war demgemäß sehr glücklich  und hat infolgedessen
alle zum Mittagsessen in der Schindler-Mensa eingeladen. 

Nach diesem enormen Meilenstein bestanden weitere Entwicklungsarbeiten, zumindest was das PCB-Design anging, nur noch in kleinen 
Modifikationen von einer Version zur anderen: manche Komponenten wurden gegen bessere und günstigere ausgetauscht, falsche Slkscreen-Texte wurden
berichtigt usw. Lediglich bei der fünften Iteration ist eine Herausforderung aufgetreten, nämlich dass der Bestand für unseren USB-Stecker
bei JLCPCB zu gering war und die Nachlieferung unsere Zeiteinschränkungen überschritten hat. Nach tagelanger Recherche wurde für eine
nachhaltigere Variante entschieden. Allerdings musste ich selber den Footprint davon erstellen, da er nicht in der Standard-KiCad-Library
vorhanden war. Aufgrund dessen, dass man im Kicads Footprint-Editor bei komplexen Geometrien im Gegensatz zu Solidworks z.B die Maßen und Abstände der Pads nicht
genau in Verhältnis zueinander setzen kann, muss über Umwege gegangen werden. Letztendlich musste ich die mechanische Zeichnung vom 
Footprint in Solidworks umsetzen, diese im Anschluss mit Inkscape bearbeiten, um sie endlich als Vorlage in Kicad zu importieren.

Die Schwierigkeiten hören noch nicht auf. Die Pads des neuen USB-Steckers sind viel dichter bevölkert als beim alten, und das hat für
einen veritablen Kampf gesorgt. Weil unser Stecker ein vertikaler und kein horizontaler ist wie die meisten USB-Stecker

\section{Hardware-Testen}

\section{Filter}