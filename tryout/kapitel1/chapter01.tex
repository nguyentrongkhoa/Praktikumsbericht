\chapter{Praktikumsbetrieb}

\section{Vorstellung des Betriebes}

Die \href{https://www.descap.de/}{DesCap GmbH} ist ein Start-up-Unternehmen mit Sitz in Berlin Alt-Mariendorf, welches sich im Bereich
der Filtergeräte mit Schwerpunkt auf Atemschutz. Zu ihren Produkten, sowohl marktreif als auch unter Entwicklung, zählen vor allem filtrienrende Gesichtsvisiere sowie 
Sensorsysteme zur Erfassung der Luftqualität und der sich darin befindenden Schadstoffe. 

\section{Weg zur Praktikumsstelle}

Auf DesCap bin ich im April 2025, also etwa sechs Monate vor Praktikumsanfang, dank einer Email von Professor Hannes 
Höppner aufmerksam geworden. Da die Stellenausschreibung mich sehr angesprochen hat, habe ich mich entschieden, meine Bewerbung
bei DesCap einzureichen. Der Bewerbungsprozess lief effizient und unkompliziert: es wurden insgesamt zwei Interviewgespräche geführt,
das erste mit Ole (einem der Gründer) und Christopher (mein künftiger Kollege); beim zweiten Gespräch hat sich Sönke,
der andere Gründer und zeitgleich Geschäftsführer, dazu gesellt. Der gesamte Bewerbungsablauf von der Bewerbung bis zur Vertragsunterzeichnung
hat ungefähr 3 Wochen gedauert, und schon am 1 Mai 2025 begann mein erster Tag als Werkstudent. Zum Anfang des Wintersemesters 2025
verwandelte sich meine Werkstudentenstelle in ein sechsmonatiges Praktikum mit Schließung des Praktikumsvertrags.
% TODO: change 480 hours to something else
Dementsprechend erstreckte es sich vom 1 Oktober 2025 bis zum 1 April 2026. Es wurden durchschnittlich 20 Wochenstunden 
erbracht, sodass im Laufe der gesamten Praktikumsdauer 480 Stunden sich angehäuft haben. 

